\documentclass[a5paper,twoside]{book}

\usepackage{tabularx}
\usepackage[pagestyles]{titlesec}
\usepackage{lipsum}
\usepackage{fontspec}
\usepackage{hyperref}
\usepackage{xcolor}
\usepackage{xstring}
\usepackage{xparse}
\usepackage{expl3}

\hypersetup{
    colorlinks,
    linkcolor={red!50!black},
    citecolor={blue!50!black},
    urlcolor={blue!80!black}
}

\renewcommand{\chaptermark}[1]{\markboth{#1}{#1}}

\newcommand{\mychaptertitle}{\ifnum\thechapter>0\chaptername\ \thechapter\ --\ \chaptertitle\fi}
\newcommand{\mysethead}[3]{\sethead[#3][#2][#1]{#1}{#2}{#3}}
\newcommand{\mysetfoot}[3]{\setfoot[#3][#2][#1]{#1}{#2}{#3}}

\newcommand{\Tstrut}{\rule{0pt}{2.6ex}}

\newfontfamily\ipafont[Path = font/, UprightFont = *-Regular]{DoulosSIL}

\newcommand{\ipa}[1]{{\ipafont #1}}

\renewpagestyle{plain}{\mysetfoot{}{}{\thepage}}
\newpagestyle{mystyle}{
    \mysethead{}{\mychaptertitle}{}
    \mysetfoot{}{}{\thepage}
}
\pagestyle{mystyle}

\titleformat{\paragraph}
{\normalfont\normalsize\itshape}{\theparagraph}{1em}{}
\titlespacing*{\paragraph}
{0pt}{3.25ex plus 1ex minus .2ex}{1.5ex plus .2ex}

\titleformat{\chapter}[display]{\normalfont\Large}{Chapter \thechapter}{1pc}{\titlerule\vspace{1pc}\Huge}


\NewDocumentEnvironment{examples}{}{
    % \ExplSyntaxOn
    % \int_new:n \count
    % \ProvideDocumentCommand{\example}{}{
    %     \paragraph*{Example \int_eval:n {\count}}
    % }
    % \ExplSyntaxOff
}{

}


\newcommand{\name}{CONLANG}
\title{\name}
\author{Einar Persson}

\begin{document}

\frontmatter
\maketitle
\tableofcontents
\mainmatter

\part{Phonology}

\chapter{Phonology}

\section{Vowels}

\newcommand{\ipaort}[2]{\ipa{#1} $\langle$#2$\rangle$}

\begin{tabular}{r|c c}
    & Front & Back \\\hline
    High     & \ipaort{i}{i} & \ipaort{u}{u}\Tstrut\\
    Mid      & \ipaort{ɛ}{e} \\
    Mid-high & \ipaort{æ}{ê} & \ipaort{ɔ}{o} \\
    Low      & \ipaort{a}{a} \\
\end{tabular}

\section{Consonants}

\begin{tabular}{r|c c c c c c c c c c}
    & Bilabial & Alveolar & Palatal & Velar \\\hline
    Nasal & \ipaort{m}{m} & \ipaort{n}{n}\Tstrut\\
    Plosive & \ipaort{p}{p} & \ipaort{t}{t} && \ipaort{k}{k}  \\
    Fricative & \ipaort{f}{f} & \ipaort{s}{s} & \ipaort{ç}{c} & \ipaort{h}{h} \\
    Approximant && \ipaort{l}{l} & \ipaort{j}{j} \\
    Rhotic && \ipaort{r}{r}
\end{tabular}

\subsection*{Non-pulmonic}
\begin{tabular}{c c c c c c}
    \ipaort{pʼ}{pp} & \ipaort{tʼ}{tt} & \ipaort{kʼ}{kk}
\end{tabular}

\section{Phonotactics}

\part{Grammar}

\newcommand{\myrepeat}[2]{
\ExplSyntaxOn
    % \ifnum 0 > {\int_eval:n #1}TRUE\fi
    \int_new:N \count
    \int_set:Nn \count {#1}
\ExplSyntaxOff
}

\NewDocumentCommand{\gloss}{m o m m}{
    \noindent\begin{quote}\begin{tabular}{l l l l l l l l l l l l l l l l}
        #1\\
        \IfNoValueTF{#2}{}{#2\\}
        #3
    \end{tabular}\\
    \noindent\begin{tabular}{l}#4\end{tabular}\end{quote}
}

\chapter{Sentence structure}

\name\ is what I call a S2-language; the subject must always be the second element.
Compare with V2-languages.\vspace{1em}

Example:

\begin{tabular}{c|c|c|c|c}
    Fundament & Subject & Verb & Object & Adjuncts\\\hline
    --- & me & et & futo \\
\end{tabular}

\begin{tabular}{c|c|c|c|c}
    Fundament & Subject & Verb & Object & Adjuncts\\\hline
    --- & me & et & futo \\
\end{tabular}

\begin{tabular}{c|c|c|c|c}
    Fundament & Subject & Verb & Object & Adjuncts\\\hline
    --- & me & et & futo \\
\end{tabular}

\chapter{Parts of Speech}

\section{Pronouns}

\begin{tabular}{r|l l l}
    & Obviate & Proximate \\\hline
    1 & me & --\Tstrut\\
    2 & to & --\\
    3 & ci & cinê
\end{tabular}

\section{Nouns}

\subsection{Cases}

\begin{tabular}{l l l l l}

\end{tabular}

\section{Verbs}

There are intransitive and transitive verbs.

The object of transitive verbs can be dropped
to turn the verb into an intransive one.

Whenever a verb is used, it's subject
becomes the \emph{theme}, and any words to refer
to it must use the \emph{proximate} gender until another \emph{theme}
has been established.

Verbs must be used in one of several moods:
\begin{itemize}
    \item Active
    \item Passive
\end{itemize}

\subsection{Active}

A verb can be marked \emph{active}.

The subject is the agent.
The object is the patient or experiencer.

\begin{examples}

    % \example
    et to futo on me
    futo to et on me

\end{examples}


\subsubsection*{Examples}

\paragraph*{Example 1}

\gloss
{Etat & me & futo}
{eat-\sc{active} & 1 & fruit}
{I eat a fruit}

\noindent \emph{me} becomes the theme.

\paragraph*{Example 2}

\gloss
{Etat & me}
{eat-\sc{active} & 1}
{I eat}

\noindent Same here: \emph{me} becomes the theme.

\subsection{Passive}

A verb can be marked \emph{passive}.

The subject is the patient or experiencer.
The second argument is the agent.

\subsubsection*{Examples}

\paragraph*{Example 1}

\gloss
{etel & futo & me}
{eat-\sc{passive} & fruit & 1}
{a fruit is eaten by me}
\vspace{0.5em}

\noindent \emph{futo} becomes the theme.

\paragraph*{Example 2}

\gloss
{etel & futo}
{eat-\sc{passive} & fruit}
{a fruit is eaten}

\noindent Same here: \emph{futo} becomes the theme.

\subsection{Incorporation}

\subsubsection{Intransitive active incorporation}

An intransitive active verb with a prepositional phrase, in the form \emph{V S prep NP},
may be transformed into \emph{prep-V S NP}, with the same meaning.

This changes the verb's valency from one to two.

\paragraph*{Example 1}

\gloss
{etat & me & a & futo}
[et--at & me & a & futo]
{eat--\sc{act} & 1 & on & fruit}
{I eat on top of a fruit}

to

\gloss
{a-etat & me & futo}
[a--et--at & me & futo]
{on--eat--\sc{act} & 1 & fruit}
{I eat on top of a fruit}

\paragraph*{Example 2}

\gloss
{etel & futo & a & me}
[et--el & futo & a & me]
{eat--\sc{pas} & fruit & on & 1}
{A fruit is eaten on me}

to

\gloss
{a-etel & futo & me}
[a--et--el & futo & me]
{on--eat--\sc{pas} & fruit & on & 1}
{A fruit is eaten on me}

\subsubsection{Transitive incorporation}

A transitive verb with a prepositional phrase, in the form \emph{V S O prep NP},
may be transformed into \emph{prep-V S NP}, with \emph{almost} the same meaning.
Note that the object is now omitted and must be inferred from context.

This doesn't change the verb's valency.

\paragraph*{Example 1}

\gloss
{etat & me & futo & a & to}
[et--at & me & futo & a & to]
{eat--\sc{act} & 1 & fruit & on & you}
{I eat a fruit on you}
to
\gloss
{a-etat & me & to}
[a--et--at & me & to]
{on--eat--\sc{act} & 1 & you}
{I eat on you}
Note that the reference to the fruit has been dropped.

\paragraph*{Example 1}

\section{Adjuncts}

\subsection{Prepositional phrases}

A prepositional phrase is built from a preposition and a noun phrase.

\subsubsection{Describing cause}

The preposition \emph{ettet} describes the cause of a verb phrase;
who or what caused the action to happen?



\part{Dictionary}


\newcommand{\replipa}[1]{%
  \saveexpandmode\noexpandarg
  \def\tempstring{#1}%
  \xStrSubstitute{\tempstring}{e}{ɛ}[\tempstring]%
  \xStrSubstitute{\tempstring}{ê}{æ}[\tempstring]%
  \xStrSubstitute{\tempstring}{o}{ɔ}[\tempstring]%
  \ipa{\tempstring}%
  \restoreexpandmode%
}
\newcommand*{\xStrSubstitute}{%
  \expandafter\StrSubstitute\expandafter%
}

\NewDocumentCommand{\entry}{m m m}{%
\textbf{#1} %\replipa{/#1/}
$\bullet$ \emph{#2} \\%
\hspace*{1cm}%
\begin{minipage}{.8\linewidth}%
    #3%
\end{minipage}\par%
}

\renewcommand{\sc}[1]{\textsc{#1}}

% \begin{twocolumn}
{
    \setlength{\parindent}{0pt}
    \chapter{\name\ → English}

    \section{A}

    \entry{a}{prep.}{On.}

    \section{E}

    \entry{et}{verb}{To eat.}

    \section{Ê}
    \section{I}
    \section{O}
    \section{U}
    \section{M}
    
    \entry{me}{pronoun}{First person pronoun.}

    \section{N}
    \section{P}
    \section{P'}
    \section{T}
    
    \entry{to}{pronoun}{Second person pronoun.}

    \section{T'}
    \section{K}
    \section{K'}
    
    \entry{k'al}{verb}{To be.}

    \section{F}
    
    \entry{futo}{noun}{Fruit.}

    \section{S}
    \section{C}
    
    \entry{ci}{pronoun}{Third person pronoun, obviate.}
    \entry{cinê}{pronoun}{Third person pronoun, proximate.}

    \section{H}
    \section{L}
    \section{J}
    \section{R}

    \chapter{English → \name}
}
% \end{twocolumn}

\end{document}